\documentclass[12pt,a4paper]{report}
\usepackage[utf8]{inputenc}
\usepackage{graphicx}

\begin{document} 


\begin{center}
    \Large{\sffamily{\textbf{NATIONAL INSTITUTE OF TECHNOLOGY}}}\\
\end{center}

\begin{center}
    \Large{\sffamily{RAIPUR , CHHATTISGARH}}\\ 
\end{center}



\begin{center}
   \huge{\texttt{TERM PAPER REPORT\\On\\ROBOTS IN HEALTH CARE SECTOR}}
  \end{center}
 
  
\begin{center}
\textbf{\underline{SUBMITTED BY:-}}\\

NAME:- MILOY KUMAR MANDAL\\
ROLL NO:- 21111032\\
BRANCH:- BIOMEDICAL ENGINEERING\\
SECTION:- A\\
SEMESTER:- 1ST\\
YEAR:- 1ST\\
EMAIL:- miloykumar@gmail.com\\

\textbf{\underline{SUBMITTED TO:-}}\\
MR.SAURABH GUPTA\\
BIOMEDICAL DEPARTMENT\\
NIT RAIPUR\\


 
\end{center} 
\clearpage

\begin{center}
  \huge{\textbf{ROBOTS IN HEALTH CARE SECTOR}}
\end{center}

\begin{center}
  \small{\textbf{Abstract}}
\end{center}

Assisting surgeries, disinfecting rooms, dispensing medication, keeping company: believe it or not these are the tasks medical robots will soon undertake in hospitals, pharmacies, or your nearest doctor’s office. These new ‘colleagues’ will definitely make a difference in every field of medicine.Emerging in the 1980s, the first robots in the medical field provided surgical assistance via robotic arm technologies. Over the years, artificial intelligence (AI)–enabled computer vision and data analytics have transformed medical robots, expanding their capabilities into many other areas of healthcare.Robots are now used not only in the operating room but also in clinical settings to support healthcare workers and enhance patient care. For example, hospitals and clinics are deploying robots for a much wider range of tasks to help reduce exposure to pathogens during the COVID-19 pandemic.The use of robotics and automation also extends to research laboratories where they are used to automate manual, repetitive, and high-volume tasks so technicians and scientists can focus their attention on more strategic tasks that make discoveries happen faster.Streamlined workflows and risk reduction provided by medical robotics offer value in many areas. For example, robots can clean and prep patient rooms independently, helping limit person-to-person contact in infectious disease wards. Robots with AI-enabled medicine identifier software reduce the time it takes to identify, match, and distribute medicine to patients in hospitals.As technologies evolve, robots will function more autonomously, eventually performing certain tasks entirely on their own. As a result, doctors, nurses, and other healthcare workers will be able to spend more time providing direct patient care.
\begin{center}
  \small{\textbf{Benefits of Robotics in Healthcare }}
\end{center}

Using robotics in the medical field enables a high level of patient care, efficient processes in clinical settings, and a safe environment for patients and healthcare workers.Medical robots support minimally invasive procedures, customized and frequent monitoring for patients with chronic diseases, intelligent therapeutics, and social engagement for elderly patients. In addition, as robots alleviate workloads, nurses and other caregivers can offer patients more empathy and human interaction, which can promote long-term well-being.Autonomous mobile robots (AMRs) simplify routine tasks, reduce the physical demands on human workers, and ensure more consistent processes. These robots can address staffing shortages and challenges by keeping track of inventory and placing timely orders to help make sure supplies, equipment, and medication are in stock where they are needed. Cleaning and disinfection AMRs enable hospital rooms to be sanitized and ready for incoming patients quickly, allowing workers to focus on patient-centric, value-driven work.To help keep healthcare workers safe, AMRs are used to transport supplies and linens in hospitals where pathogen exposure is a risk. Cleaning and disinfection robots limit pathogen exposure while helping reduce hospital acquired infections (HAIs)—and hundreds of healthcare facilities are already using them1. Social robots, a type of AMR, also help with heavy lifting, such as moving beds or patients, which reduces physical strain on healthcare workers.As motion control technologies have advanced, surgical-assistance robots have become more precise. These robots help surgeons achieve new levels of speed and accuracy while performing complex operations with AI- and computer vision‒capable technologies. Some surgical robots may even be able to complete tasks autonomously, allowing surgeons to oversee procedures from a console.\par

Surgeries performed with robotics assistance fall into two main categories:-\par
1. Minimally invasive surgeries for the torso. These include robotic hysterectomy, robotic prostatectomy, bariatric surgery, and other procedures primarily focused on soft tissues. After insertion through a small incision, these robots lock themselves into place, creating a stable platform from which to perform surgeries via remote control. Open surgery using large incisions was once the norm for most internal procedures. Recovery times were much longer, and the potential for infection and other complications was greater. Working manually through a button-sized incision is extremely difficult, even for an experienced surgeon. Surgical robots make these procedures easy and accurate, with a goal to reduce infections and other complications.\par

2.Orthopedic surgeries. Devices can be preprogrammed to perform common orthopedic surgeries, such as knee and hip replacements. Combining smart robotic arms, 3D imaging, and data analytics, these robots enable more predictable results by employing spatially defined boundaries to assist the surgeon. AI modeling enables robots to be trained in specific orthopedic surgeries, with precise direction for where to go and how to perform the procedures.\par

The ability to share a video feed from the operating room to other locations—near or far—allows surgeons to benefit from consultations with other specialists in their field. As a result, patients have the best surgeons involved in their procedures.The field of surgical robotics is evolving to make greater use of AI. Computer vision enables surgical robots to differentiate between types of tissue within their field of view. For example, surgical robots now have the ability to help surgeons avoid nerves and muscles during procedures2. High-definition 3D computer vision can provide surgeons with detailed information and enhanced performance during procedures. Eventually, robots will be able to take over small subprocedures, such as suturing or other defined tasks, under the watchful gaze of the surgeon.Robotics also plays a key role in surgeon education. Simulation platforms use AI and virtual reality to provide surgical robotics training. Within the virtual environment, surgeons can practice procedures and hone skills using robotics controls.\par
\begin{center}
  \small{\textbf{Modular Robots}}
\end{center}
Modular robots enhance other systems and can be configured to perform multiple functions. In healthcare, these include therapeutic exoskeleton robots and prosthetic robotic arms and legs.Therapeutic robots can help with rehabilitation after strokes, paralysis, or traumatic brain injuries or with impairments caused by multiple sclerosis. A wheelchair-mounted robotic arm currently being developed by Intel and Accenture aims to assist patients with spinal injuries in performing daily tasks. When robots are equipped with AI and depth cameras, they can monitor a patient’s form as they go through prescribed exercises, measuring degrees of motion in different positions and tracking progress more precisely than the human eye. They can also interact with patients to provide coaching and encouragement.\par

\begin{center}
  \small{\textbf{Surgical Robots}}
\end{center}

Major manufacturers are increasing their R and D efforts within robotic surgical systems. The overall market is currently dominated by Intuitive Surgical, but the landscape is rapidly changing. The entrance of major manufacturers such as Johnson  and  Johnson and Medtronicare is bolstering the medtech surgical robotics market.There are specific product lines from each company focusing on individual therapeutic areas for minimally invasive robotic surgery. For example, the da Vinci System is a general surgical robot focusing on a myriad of surgical procedures in urological, bariatric, and gynaecological surgical procedures. Additionally, the MAKO System from Stryker specialises in orthopaedic surgery, specifically partial and complete knee replacements.The key to market domination will be product proliferation, as each company tries to highlight its own features. Specific companies have remarkably distinct operation methods, such as seen with Intuitive Surgical compared to TransEnterix. Both companies offer robotic surgical systems with system specific attachments, but Intuitive has built-in chips to determine the use of their disposable accessories, and TransEnterix’s attachments are reusable.\par

\begin{center}
  \small{\textbf{Care Robots}}
\end{center}

The number of robots used to provide care and support to elderly and disabled patients is currently very low, but is expected to increase significantly over the next decade, particularly in countries like Japan, which is facing a predicted shortfall in the number of available caregivers. Initial use cases for these products are relatively simple, such as helping people get into and out of bed, but they will increasingly be called upon to perform more complex tasks, from reminding patients when to take medication to providing emotional support and interaction for those lacking regular human contact.Another expected use case for care robots is to assist nurses with the multitude of tasks that they perform on an hourly basis. Many of these tasks are simple but vital, such as taking blood, recording temperature, or improving patient hygiene. If robots were able to help with these simple repetitive tasks, it would give nurses more time to focus on individualised patient care and devising treatment plans. Products like the Robear Japanese, developed by research institute RIKEN and Sumitomo Riko, are already assisting patients and nurses in Japan.Toyota and Honda have been developing human support robots (HSRs) for many years. In 2016, Toyota launched a 1bn US dollar five-year project to open and run two AI/robotics labs in Palo Alto, California, US under the leadership of former Defense Advanced Research Projects Agency (DARPA) robotics chief Gill Pratt. The facilities were aimed as much at Toyota’s HSR division as its automotive operation. Honda is doing something similar, but the project is based in Tokyo.AIST’s Paro is classified as a therapeutic robot. Designed to be cute and elicit an emotional response from patients in hospitals and nursing homes, Paro is a robotic baby harp seal covered in soft white fur and exhibits many of the same behaviours as a real pet.

\begin{center}
  \small{\textbf{Autonomous Mobile Robots}}
\end{center}

Healthcare organizations often rely on AMRs because of their ability to assist with critical needs such as disinfection, telepresence, and delivery of medication and medical supplies, creating safe environments while freeing up staff to spend more time with patients. When equipped with light detection and ranging (LiDAR) systems, visual compute, or mapping capabilities, AMRs can self-navigate to patients in exam or hospital rooms, allowing clinicians to interact from afar. If an AMR is controlled by a remote specialist or other worker, it can accompany doctors as they make hospital rounds, allowing a specialist to contribute via an on-screen consultation regarding patient diagnostics and care.Some robots can assist professionals before a patient is checked in. For example, one autonomous robot in Mexico, developed by start-up Roomie, is being used to help medical staff with high-risk COVID-19 patients. Named RoomieBot, it triages patients by taking their temperature, blood oxygen level, and medical history when they arrive at the hospital. RoomieBot uses Intel-based technology, including AI algorithms that run on the Intel Movidius Vision Processing Unit (VPU), 8th Gen Intel NUCs, and Intel RealSense cameras.\par
 
\begin{center}
  \small{\textbf{Social Robots}}
\end{center}

Social robots interact directly with humans. These “friendly” robots can be used in long-term care environments to provide social interaction and monitoring. They may encourage patients to comply with treatment regimens or provide cognitive engagement, helping to keep patients alert and positive. They can also be used to offer directions to visitors and patients inside the hospital environment. In general, social robots help reduce caregiver workloads and improve patients’ emotional well-being.With the advancement of robotics and artificial intelligence, social companion robots started to take shape: these human or animal-shaped, smaller or bigger mechanic creatures are able to carry out different tasks and have interactions with humans and their environment. In the future, they might become every parent’s little helper in the kitchen, might support the guard dog in keeping the house safe, might teach the children and be their companion and support the elderly from reminding them to take their medication until keeping them company when they feel lonely.Jibo, Pepper, Paro, Zora, and Buddy are all existing examples for caring social companion robots. Some of them even have touch sensors, cameras, and microphones, thus their owners can get into discussions with them, ask them to find a great concert for that night or just remind them about their medications.\par

\begin{center}
  \small{\textbf{Nanorobots swimming in blood}}
\end{center}

While we have definitely not reached the era of nanotechnology, trends point towards the technology becoming more and more significant. With the emergence of digestibles and digital pills, we get closer to nanorobots step by step. On that front: researchers from the Max Planck Institute have been experimenting with exceptionally micro-sized – smaller than a millimeter – robots that literally swim through your bodily fluids and could be used to deliver drugs or other medical relief in a highly-targeted way. These scallop-like microbots are designed to swim through non-Newtonian fluids, like your bloodstream, around your lymphatic system, or across the slippery goo on the surface of your eyeballs.The origami robot, despite its size, is just as impressive as a super-strong carrier one. When swallowed, the capsule containing it dissolves in the patient’s stomach and unfolds itself. Controlled by a technician with the help of magnetic fields it can patch up wounds in the stomach lining or safely remove foreign items such as swallowed toys.




\begin{center}
  \huge{\textbf{The Future of Robotics in the Medical Field}}
\end{center}

As machine learning, data analytics, computer vision, and other technologies advance, medical robotics will evolve to complete tasks autonomously and more efficiently and accurately. Technology companies are working in collaboration with technology providers and researchers to explore the next generation of robotics solutions. By providing technology and research support, Intel is helping drive the discovery of new applications for AI and IoT technologies within the field of medical robotics. These contributions support ongoing innovations that increase automation, drive efficiencies, and solve some of the greatest healthcare challenges.Health robotics will continue to evolve alongside advancements in machine learning, data analytics, computer vision, and other technologies. Robots of all types will continue to evolve to complete tasks autonomously, efficiently, and accurately.\par


Small and big, smiley and faceless, surgical and pharma dispensing ones: robotics moves in all shapes and forms with big leaps into healthcare. That might be scary for many, but they have the potential to do good: to bring medical care to regions where there is none to be found; to make the production and distribution of pharmaceuticals cheaper and more efficient; to lighten the load of medical professionals; to help people walk again.To reap the benefits and avoid the potential dangers of such a technological revolution, we need to keep ourselves informed about the strides that science makes so that we can better prepare and adapt to the not-so-distant future where medical robots play a crucial role and work closely with us.\par


\begin{center}
  \huge{\textbf{REFRENCES}}
\end{center}

1. https://www.intel.com/content/www/us/en/healthcare-it/robotics-in-healthcare.html\par

2. https://medicalfuturist.com/robotics-healthcare/ \par

3.https://www.medicaldevice-network.com/comment/what-are-the-main-types-of-robots-used-in-healthcare/ \par

4.https://blog.robotiq.com/5-ways-cobots-are-used-in-medicine-and-healthcare \par

\end{document}